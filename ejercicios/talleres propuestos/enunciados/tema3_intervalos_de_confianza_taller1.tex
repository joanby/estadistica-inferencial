% Options for packages loaded elsewhere
\PassOptionsToPackage{unicode}{hyperref}
\PassOptionsToPackage{hyphens}{url}
\PassOptionsToPackage{dvipsnames,svgnames*,x11names*}{xcolor}
%
\documentclass[
]{article}
\usepackage{lmodern}
\usepackage{amssymb,amsmath}
\usepackage{ifxetex,ifluatex}
\ifnum 0\ifxetex 1\fi\ifluatex 1\fi=0 % if pdftex
  \usepackage[T1]{fontenc}
  \usepackage[utf8]{inputenc}
  \usepackage{textcomp} % provide euro and other symbols
\else % if luatex or xetex
  \usepackage{unicode-math}
  \defaultfontfeatures{Scale=MatchLowercase}
  \defaultfontfeatures[\rmfamily]{Ligatures=TeX,Scale=1}
\fi
% Use upquote if available, for straight quotes in verbatim environments
\IfFileExists{upquote.sty}{\usepackage{upquote}}{}
\IfFileExists{microtype.sty}{% use microtype if available
  \usepackage[]{microtype}
  \UseMicrotypeSet[protrusion]{basicmath} % disable protrusion for tt fonts
}{}
\makeatletter
\@ifundefined{KOMAClassName}{% if non-KOMA class
  \IfFileExists{parskip.sty}{%
    \usepackage{parskip}
  }{% else
    \setlength{\parindent}{0pt}
    \setlength{\parskip}{6pt plus 2pt minus 1pt}}
}{% if KOMA class
  \KOMAoptions{parskip=half}}
\makeatother
\usepackage{xcolor}
\IfFileExists{xurl.sty}{\usepackage{xurl}}{} % add URL line breaks if available
\IfFileExists{bookmark.sty}{\usepackage{bookmark}}{\usepackage{hyperref}}
\hypersetup{
  pdftitle={Ejercicios Tema 3 - Intervalos de Confianza. Taller 1},
  pdfauthor={Ricardo Alberich, Juan Gabriel Gomila y Arnau Mir},
  colorlinks=true,
  linkcolor=red,
  filecolor=Maroon,
  citecolor=blue,
  urlcolor=blue,
  pdfcreator={LaTeX via pandoc}}
\urlstyle{same} % disable monospaced font for URLs
\usepackage[margin=1in]{geometry}
\usepackage{graphicx,grffile}
\makeatletter
\def\maxwidth{\ifdim\Gin@nat@width>\linewidth\linewidth\else\Gin@nat@width\fi}
\def\maxheight{\ifdim\Gin@nat@height>\textheight\textheight\else\Gin@nat@height\fi}
\makeatother
% Scale images if necessary, so that they will not overflow the page
% margins by default, and it is still possible to overwrite the defaults
% using explicit options in \includegraphics[width, height, ...]{}
\setkeys{Gin}{width=\maxwidth,height=\maxheight,keepaspectratio}
% Set default figure placement to htbp
\makeatletter
\def\fps@figure{htbp}
\makeatother
\setlength{\emergencystretch}{3em} % prevent overfull lines
\providecommand{\tightlist}{%
  \setlength{\itemsep}{0pt}\setlength{\parskip}{0pt}}
\setcounter{secnumdepth}{5}
\renewcommand{\contentsname}{Contenidos}

\title{Ejercicios Tema 3 - Intervalos de Confianza. Taller 1}
\author{Ricardo Alberich, Juan Gabriel Gomila y Arnau Mir}
\date{Curso completo de estadística inferencial con R y Python}

\begin{document}
\maketitle

{
\hypersetup{linkcolor=blue}
\setcounter{tocdepth}{2}
\tableofcontents
}
\hypertarget{estimaciuxf3n-por-intervalos-taller-1}{%
\section{Estimación por intervalos taller
1}\label{estimaciuxf3n-por-intervalos-taller-1}}

\hypertarget{ejercicio-1}{%
\subsection{Ejercicio 1}\label{ejercicio-1}}

De una población de barras de hierro se extrae una muestra de \(64\)
barras y se calcula la resistencia a la rotura por tracción se obtiene
que \(\overline{X}=1012\ Kg/cm^{2}\). Se sabe por experiencia que en
este tipo de barras \(\sigma=25\). Calcular un intervalo de confianza
para \(\mu\) al nivel 0.95.

\sol{$\left(1005.88 , 1018.13\right)$}

\hypertarget{ejercicio-2}{%
\subsection{Ejercicio 2}\label{ejercicio-2}}

Para investigar el C.I. medio de una cierta población de estudiantes, se
realiza un test a \(400\) estudiantes. La media y la desviación típica
muestrales obtenidas son \(\overline{x}=86\) y \(\tilde{s}_{X}=10.2\).
Calcular un intervalo para \(\mu\) con un nivel de significación del
98\%. \sol{$\left(84.8117, 87.1883\right)$}

\hypertarget{ejercicio-3}{%
\subsection{Ejercicio 3}\label{ejercicio-3}}

Para investigar un nuevo tipo de combustible para cohetes espaciales, se
disparan cuatro unidades y se miden las velocidades iniciales. Los
resultados obtenidos, expresados en Km/h, son :19600, 20300, 20500,
19800. Calcular un intervalo para la velocidad media \(\mu\) con un
nivel de confianza del 95\%, suponiendo que las velocidades son
normales. \sol{$\left(19381.7, 20718.3\right)$} 20718.3

\hypertarget{ejercicio-4}{%
\subsection{Ejercicio 4}\label{ejercicio-4}}

Un fabricante de cronómetros quiere calcular un intervalo de estimación
de la desviación típica del tiempo marcado en \(100\) horas por todos
los cronómetros de un cierto modelo. Para ello pone en marcha \(10\)
cronómetros del modelo durante \(100\) horas y encuentra que
\(\tilde{s}_{X}=50\) segundos. Encontrar un intervalo para el parámetro
\(\sigma^2\) con \(\alpha=0.01\), suponiendo que la población del tiempo
marcado por los cronómetros es normal.
\sol{$\left(953.834,12968.3\right)$} \textbackslash end\{prob\}

\hypertarget{ejercicio-5}{%
\subsection{Ejercicio 5}\label{ejercicio-5}}

Un auditor informático quiere investigar la proporción de rutinas de un
programa que presentan una determinada irregularidad. Para ello observa
\(120\) rutinas, resultando que \(30\) de ellas presentan alguna
irregularidad. Con estos datos buscar unos límites de confianza para la
proporción \(p\) de rutinas de la población que presentan esa
irregularidad con probabilidad del 95\%.

\sol{$\left(0.1725,0.3275\right)$}

\hypertarget{ejercicio-6}{%
\subsection{Ejercicio 6}\label{ejercicio-6}}

En una ciudad A de 400 propietarios de coches 125 tienen una marca
\(X\); y en otra población B, de 600 propietarios, 180 tienen la marca
\(X\). Calcular un intervalo de confianza del 95\% para la diferencia de
proporciones entre la ciudad A y la B.
\sol{ $\left(-0.111,0.136\right)$}

\hypertarget{ejercicio-7}{%
\subsection{Ejercicio 7}\label{ejercicio-7}}

Una infección por un virus puede haber perjudicado a muchos ordenadores
con \emph{Windwos}. Desde el Centro de Alerta Temprana (CAT) se quiere
calcular la proporción de ordenadores infectados. El jefe del centro os
pide que calculéis el tamaño de una muestra para que el intervalo de
confianza de la proporción muestral de ordenadores infectados tenga
amplitud de a lo sumo \(0.01\) con una probabilidad del 90\%.

\hypertarget{ejercicio-7-1}{%
\subsection{Ejercicio 7}\label{ejercicio-7-1}}

Se han medido los siguientes valores (en miles de personas) para la
audiencia de un programa de televisión en distintos días (supuestos
igualmente distribuidos e independientes):

\[521, 742, 593, 635, 788, 717, 606, 639, 666, 624.\]

Construir un intervalo de confianza del \(90\)\%, para la audiencia
poblacional media y otro para la varianza poblacional, bajo la hipótesis
de que la población de audiencias sigue una ley normal.

\hypertarget{ejercicio-8}{%
\subsection{Ejercicio 8}\label{ejercicio-8}}

Supongamos que la empresa para la que trabajamos está en un proyecto de
investigación, financiado con fondos de la Comunidad Europea, que
pretende extender una nueva aplicación de las TIC. Una de las tareas del
proyecto es realizar una encuesta de opinión sobre el grado de
aceptación que tendría esta nueva tecnología en el mercado europeo. De
entre todas las universidades y empresas participantes en el proyecto,
es a tu empresa a la que le toca hacer el protocolo de la encuesta,
llevarla a cabo y redactar esta parte del informe final. Como eres el
último que llegó a la empresa y el resto de miembros del equipo no se
acuerda de la estadística que vio en la carrera, te toca a ti cargar con
la responsabilidad. Claro que el coste de la encuesta depende del número
\(n\) de entrevistas que se realicen y el error de las proporciones de
las contestaciones disminuye cuando \(n\) aumenta. Como no sabes cuánto
dinero está dispuesto a gastar tu jefe, tabula los tamaños muestrales
para los errores \(\pm 5\%, \pm 3\%, \pm 2\%,\pm 1\%\), y para niveles
de confianza \(0.95\) y \(0.99\), suponiendo el peor caso. Añade un
comentario para que el equipo de dirección del proyecto, en el que hay
componentes ignorantes en materia de encuestas, vea como quedarían
redactado los datos técnicos de la encuesta, y pueda decidir el tamaño
muestral leyendo tu informe.

\hypertarget{ejercicio-9}{%
\subsection{Ejercicio 9}\label{ejercicio-9}}

El número de reservas semanales de billetes de cierto vuelo de una
compañía aérea sigue una distribución aproximadamente normal. Se toma un
muestra aleatoria de \(81\) observaciones de números de reservas de este
vuelo: el número medio de reservas muestral resulta ser \(112\),
mientras que la desviación típica muestral es \(36\). Además de estos
\(81\) vuelos, \(30\) llegaron a su destino con un retraso de más de
\(15\) minutos.

\begin{enumerate}
\def\labelenumi{\arabic{enumi}.}
\tightlist
\item
  Calcular un intervalo de confianza del \(95\%\) para el número medio
  poblacional de reservas en este vuelo.
\item
  Calcular un intervalo de confianza de \(95\%\) para la varianza
  poblacional de las reservas.
\item
  Calcular un intervalo de confianza del \(95\%\) para la proporción
  poblacional de vuelos que llegan con un retraso de más de \(15\)
  minutos.
\item
  Calcular el tamaño muestral que asegura un intervalo de confianza de
  amplitud \(0.1\) para la proporción de vuelos que llegan con un
  retraso de más de \(15\) minutos al nivel de confianza \(95\%\).
  \sol{a) $\left(104.16,119.84)\right)$; b) $\left(972.343,1814.08)\right)$; c)
  $\left(0.265,0.475)\right)$; d) $n=385$}
\end{enumerate}

\hypertarget{ejercicio-10}{%
\subsection{Ejercicio 10}\label{ejercicio-10}}

Una empresa cervecera sabe que las cantidades de cerveza que contienen
sus latas sigue una distribución normal con desviación típica
poblacional \(0.03\) litros.

\begin{enumerate}
\def\labelenumi{\arabic{enumi}.}
\tightlist
\item
  Se extrae una muestra aleatoria de \(25\) latas y, a partir de la
  misma, un experto en estadística construye un intervalo de confianza
  para la media poblacional del contenido en litros de las latas que
  discurre entre \(0.32\) y \(0.34\) ¿Cuál es el nivel de confianza de
  este intervalo?
\item
  Un gerente de esta empresa exige un intervalo de confianza del
  \(99\%\) que tenga una amplitud máxima de \(0.03\) litros a cada lado
  de la media muestral ¿Cuántas observaciones son necesarias, como
  mínimo, para alcanzar este objetivo? \sol{a) $90.3\%$; b) $n=7$}
\end{enumerate}

\end{document}
