% Options for packages loaded elsewhere
\PassOptionsToPackage{unicode}{hyperref}
\PassOptionsToPackage{hyphens}{url}
\PassOptionsToPackage{dvipsnames,svgnames*,x11names*}{xcolor}
%
\documentclass[
]{article}
\usepackage{lmodern}
\usepackage{amssymb,amsmath}
\usepackage{ifxetex,ifluatex}
\ifnum 0\ifxetex 1\fi\ifluatex 1\fi=0 % if pdftex
  \usepackage[T1]{fontenc}
  \usepackage[utf8]{inputenc}
  \usepackage{textcomp} % provide euro and other symbols
\else % if luatex or xetex
  \usepackage{unicode-math}
  \defaultfontfeatures{Scale=MatchLowercase}
  \defaultfontfeatures[\rmfamily]{Ligatures=TeX,Scale=1}
\fi
% Use upquote if available, for straight quotes in verbatim environments
\IfFileExists{upquote.sty}{\usepackage{upquote}}{}
\IfFileExists{microtype.sty}{% use microtype if available
  \usepackage[]{microtype}
  \UseMicrotypeSet[protrusion]{basicmath} % disable protrusion for tt fonts
}{}
\makeatletter
\@ifundefined{KOMAClassName}{% if non-KOMA class
  \IfFileExists{parskip.sty}{%
    \usepackage{parskip}
  }{% else
    \setlength{\parindent}{0pt}
    \setlength{\parskip}{6pt plus 2pt minus 1pt}}
}{% if KOMA class
  \KOMAoptions{parskip=half}}
\makeatother
\usepackage{xcolor}
\IfFileExists{xurl.sty}{\usepackage{xurl}}{} % add URL line breaks if available
\IfFileExists{bookmark.sty}{\usepackage{bookmark}}{\usepackage{hyperref}}
\hypersetup{
  pdftitle={Ejercicios Tema 4 - Contraste hipótesis. Taller 3},
  pdfauthor={Ricardo Alberich, Juan Gabriel Gomila y Arnau Mir},
  colorlinks=true,
  linkcolor=red,
  filecolor=Maroon,
  citecolor=blue,
  urlcolor=blue,
  pdfcreator={LaTeX via pandoc}}
\urlstyle{same} % disable monospaced font for URLs
\usepackage[margin=1in]{geometry}
\usepackage{color}
\usepackage{fancyvrb}
\newcommand{\VerbBar}{|}
\newcommand{\VERB}{\Verb[commandchars=\\\{\}]}
\DefineVerbatimEnvironment{Highlighting}{Verbatim}{commandchars=\\\{\}}
% Add ',fontsize=\small' for more characters per line
\usepackage{framed}
\definecolor{shadecolor}{RGB}{248,248,248}
\newenvironment{Shaded}{\begin{snugshade}}{\end{snugshade}}
\newcommand{\AlertTok}[1]{\textcolor[rgb]{0.94,0.16,0.16}{#1}}
\newcommand{\AnnotationTok}[1]{\textcolor[rgb]{0.56,0.35,0.01}{\textbf{\textit{#1}}}}
\newcommand{\AttributeTok}[1]{\textcolor[rgb]{0.77,0.63,0.00}{#1}}
\newcommand{\BaseNTok}[1]{\textcolor[rgb]{0.00,0.00,0.81}{#1}}
\newcommand{\BuiltInTok}[1]{#1}
\newcommand{\CharTok}[1]{\textcolor[rgb]{0.31,0.60,0.02}{#1}}
\newcommand{\CommentTok}[1]{\textcolor[rgb]{0.56,0.35,0.01}{\textit{#1}}}
\newcommand{\CommentVarTok}[1]{\textcolor[rgb]{0.56,0.35,0.01}{\textbf{\textit{#1}}}}
\newcommand{\ConstantTok}[1]{\textcolor[rgb]{0.00,0.00,0.00}{#1}}
\newcommand{\ControlFlowTok}[1]{\textcolor[rgb]{0.13,0.29,0.53}{\textbf{#1}}}
\newcommand{\DataTypeTok}[1]{\textcolor[rgb]{0.13,0.29,0.53}{#1}}
\newcommand{\DecValTok}[1]{\textcolor[rgb]{0.00,0.00,0.81}{#1}}
\newcommand{\DocumentationTok}[1]{\textcolor[rgb]{0.56,0.35,0.01}{\textbf{\textit{#1}}}}
\newcommand{\ErrorTok}[1]{\textcolor[rgb]{0.64,0.00,0.00}{\textbf{#1}}}
\newcommand{\ExtensionTok}[1]{#1}
\newcommand{\FloatTok}[1]{\textcolor[rgb]{0.00,0.00,0.81}{#1}}
\newcommand{\FunctionTok}[1]{\textcolor[rgb]{0.00,0.00,0.00}{#1}}
\newcommand{\ImportTok}[1]{#1}
\newcommand{\InformationTok}[1]{\textcolor[rgb]{0.56,0.35,0.01}{\textbf{\textit{#1}}}}
\newcommand{\KeywordTok}[1]{\textcolor[rgb]{0.13,0.29,0.53}{\textbf{#1}}}
\newcommand{\NormalTok}[1]{#1}
\newcommand{\OperatorTok}[1]{\textcolor[rgb]{0.81,0.36,0.00}{\textbf{#1}}}
\newcommand{\OtherTok}[1]{\textcolor[rgb]{0.56,0.35,0.01}{#1}}
\newcommand{\PreprocessorTok}[1]{\textcolor[rgb]{0.56,0.35,0.01}{\textit{#1}}}
\newcommand{\RegionMarkerTok}[1]{#1}
\newcommand{\SpecialCharTok}[1]{\textcolor[rgb]{0.00,0.00,0.00}{#1}}
\newcommand{\SpecialStringTok}[1]{\textcolor[rgb]{0.31,0.60,0.02}{#1}}
\newcommand{\StringTok}[1]{\textcolor[rgb]{0.31,0.60,0.02}{#1}}
\newcommand{\VariableTok}[1]{\textcolor[rgb]{0.00,0.00,0.00}{#1}}
\newcommand{\VerbatimStringTok}[1]{\textcolor[rgb]{0.31,0.60,0.02}{#1}}
\newcommand{\WarningTok}[1]{\textcolor[rgb]{0.56,0.35,0.01}{\textbf{\textit{#1}}}}
\usepackage{graphicx}
\makeatletter
\def\maxwidth{\ifdim\Gin@nat@width>\linewidth\linewidth\else\Gin@nat@width\fi}
\def\maxheight{\ifdim\Gin@nat@height>\textheight\textheight\else\Gin@nat@height\fi}
\makeatother
% Scale images if necessary, so that they will not overflow the page
% margins by default, and it is still possible to overwrite the defaults
% using explicit options in \includegraphics[width, height, ...]{}
\setkeys{Gin}{width=\maxwidth,height=\maxheight,keepaspectratio}
% Set default figure placement to htbp
\makeatletter
\def\fps@figure{htbp}
\makeatother
\setlength{\emergencystretch}{3em} % prevent overfull lines
\providecommand{\tightlist}{%
  \setlength{\itemsep}{0pt}\setlength{\parskip}{0pt}}
\setcounter{secnumdepth}{5}
\renewcommand{\contentsname}{Contenidos}

\title{Ejercicios Tema 4 - Contraste hipótesis. Taller 3}
\author{Ricardo Alberich, Juan Gabriel Gomila y Arnau Mir}
\date{Curso completo de estadística inferencial con R y Python}

\begin{document}
\maketitle

{
\hypersetup{linkcolor=blue}
\setcounter{tocdepth}{2}
\tableofcontents
}
\hypertarget{contraste-hipuxf3tesis-taller-3-contrastes-de-dos-paruxe1metros.}{%
\section{Contraste hipótesis taller 3: Contrastes de dos
parámetros.}\label{contraste-hipuxf3tesis-taller-3-contrastes-de-dos-paruxe1metros.}}

\textbf{Comparación de medias.}

\hypertarget{ejercicio-1}{%
\subsection{Ejercicio 1}\label{ejercicio-1}}

Para comparar la producción media de dos procedimientos de fabricación
de cierto producto se toman dos muestras, una con la cantidad producida
durante 25 días con el primer método y otra con la cantidad producida
durante 16 días con el segundo método. Por experiencia se sabe que la
varianza del primer procedimiento es \(\sigma_{1}^2=12\) y al del
segundo \(\sigma_{2}^2=10\). De las muestras obtenemos que
\(\overline{X}_{1}=136\) para el primer procedimiento y
\(\overline{X}_{2}=128\) para el segundo. Si \(\mu_{1}\) y \(\mu_{2}\)
son los valores esperados para cada uno de los procedimientos, calcular
un intervalo de confianza para \(\mu_{1}-\mu_{2}\) al nivel 99\%. \%
sol1\{\(\left(5.2989,10.7011\right)\)\}

\hypertarget{ejercicio-2}{%
\subsection{Ejercicio 2}\label{ejercicio-2}}

Estamos interesados en comparar la vida media, expresada en horas de dos
tipos de componentes electrónicos. Para ello se toma una muestra de cada
tipo y se obtiene:

\begin{center}
\begin{tabular}{|c|c|c|c|}
\hline Tipo & tamaño & $\overline{x}$ & $\tilde{s}$\\ \hline \hline 1 & 50 & 1260 & 20\\ \hline 2 &
100 & 1240 & 18\\ \hline
\end{tabular}
\end{center}

Calcular un intervalo de confianza para \(\mu_{1}-\mu_{2}\) (\(\mu_{1}\)
esperanza del primer grupo y \(\mu_{2}\) esperanza del segundo grupo) al
nivel 98\% Suponer si es necesario las poblaciones aproximadamente
normales. \% sol1\{\(\left(12.19,27.81\right)\)\}

\hypertarget{ejercicio-3}{%
\subsection{Ejercicio 3}\label{ejercicio-3}}

Para reducir la concentración de ácido úrico en la sangre se prueban dos
drogas. La primera se aplica a un grupo de 8 pacientes y la segunda a un
grupo de 10. Las disminuciones observadas en las concentraciones de
ácido úrico de los distintos pacientes expresadas en tantos por cien de
concentración después de aplicado el tratamiento son:

\begin{center}
\begin{tabular}{|c|c|c|c|c|c|c|c|c|c|c|}
droga 1 & 20 & 12 & 16 & 18 & 13 & 22 & 15 & 20\\ \hline droga 2 & 17 & 14 & 12 & 10 & 15 &
13 & 9 & 19 & 20 & 11
\end{tabular}
\end{center}

Suponer que las reducciones de ácido úrico siguen una distribución
normal son independientes

Contrastar la igualdad de medias contra que la droga 1 es mejor (menor
media) que la droga 2. Resolver el test en los dos casos varianzas
iguales y varianzas distintas. Calcular el intervalo de confianza
asociado al contraste.

\hypertarget{ejercicio-4}{%
\subsection{Ejercicio 4}\label{ejercicio-4}}

Para comparar la dureza media de dos tipos de aleaciones (tipo 1 y tipo
2) se hacen 20 pruebas de dureza con la de tipo 1 y 25 con la de tipo 2.
Obteniéndose los resultados siguientes:

\begin{Shaded}
\begin{Highlighting}[]
\KeywordTok{set.seed}\NormalTok{(}\DecValTok{345}\NormalTok{)}
\NormalTok{aleacion1=}\KeywordTok{round}\NormalTok{(}\FloatTok{0.2}\OperatorTok{*}\NormalTok{(}\KeywordTok{rnorm}\NormalTok{(}\DecValTok{20}\NormalTok{))}\OperatorTok{+}\FloatTok{18.2}\NormalTok{,}\DecValTok{2}\NormalTok{)}
\NormalTok{aleacion2=}\KeywordTok{round}\NormalTok{(}\FloatTok{0.5}\OperatorTok{*}\NormalTok{(}\KeywordTok{rnorm}\NormalTok{(}\DecValTok{25}\NormalTok{))}\OperatorTok{+}\FloatTok{17.8}\NormalTok{,}\DecValTok{2}\NormalTok{)}
\end{Highlighting}
\end{Shaded}

\begin{Shaded}
\begin{Highlighting}[]
\NormalTok{aleacion1=}\KeywordTok{c}\NormalTok{(}\FloatTok{18.04}\NormalTok{,}\FloatTok{18.14}\NormalTok{,}\FloatTok{18.17}\NormalTok{,}\FloatTok{18.14}\NormalTok{,}\FloatTok{18.19}\NormalTok{,}\FloatTok{18.07}\NormalTok{,}\FloatTok{18.01}\NormalTok{,}\FloatTok{18.54}\NormalTok{,}
            \FloatTok{18.53}\NormalTok{,}\FloatTok{18.56}\NormalTok{,}\FloatTok{18.57}\NormalTok{,}\FloatTok{17.92}\NormalTok{,}\FloatTok{18.03}\NormalTok{,}\FloatTok{18.26}\NormalTok{,}\FloatTok{18.38}\NormalTok{,}\FloatTok{17.92}\NormalTok{,}
            \FloatTok{18.31}\NormalTok{,}\FloatTok{18.41}\NormalTok{,}\DecValTok{18}\NormalTok{,}\FloatTok{18.26}\NormalTok{)}
\NormalTok{aleacion2=}\KeywordTok{c}\NormalTok{(}\FloatTok{18.14}\NormalTok{,}\FloatTok{17.87}\NormalTok{,}\FloatTok{17.53}\NormalTok{,}\FloatTok{17.61}\NormalTok{,}\FloatTok{17.24}\NormalTok{,}\FloatTok{18.01}\NormalTok{,}\FloatTok{17.04}\NormalTok{,}\FloatTok{17.82}\NormalTok{,}
            \FloatTok{17.97}\NormalTok{,}\FloatTok{17.75}\NormalTok{,}\FloatTok{18.48}\NormalTok{,}\FloatTok{17.34}\NormalTok{,}\FloatTok{17.29}\NormalTok{,}\FloatTok{16.81}\NormalTok{,}\FloatTok{19.21}\NormalTok{,}\FloatTok{17.6}\NormalTok{,}
            \FloatTok{17.85}\NormalTok{,}\FloatTok{17.98}\NormalTok{,}\FloatTok{18.04}\NormalTok{,}\FloatTok{18.13}\NormalTok{,}\FloatTok{18.26}\NormalTok{,}\FloatTok{18.1}\NormalTok{,}\FloatTok{18.37}\NormalTok{,}\FloatTok{17.48}
\NormalTok{            ,}\FloatTok{17.58}\NormalTok{)}
\end{Highlighting}
\end{Shaded}

\[\overline{X}_{1}=18.2,\quad S_{1}=0.2 \mbox{ y}\]

\[\overline{X}_{2}=17.8;\quad S_{2}=0.5\]

Suponer que la población de las durezas es normal y que las desviaciones
típicas no son iguales. Hacer lo mismo si las varianzas son distintas.

\hypertarget{ejercicio-5}{%
\subsection{Ejercicio 5}\label{ejercicio-5}}

Se encuestó a dos muestras independientes de empresas, en las islas de
Ibiza y otra en Mallorca, sobre si utilizaban sistemas de almacenamiento
en la nube. La encuesta de Ibiza tuvo un tamaño \(n_1=500\) y \(200\)
usuarios mientras que en Mallorca se encuestaron a \(n_2=750\) y se
obtuvo un resultado de \(210\) usuarios.

se pide:

\begin{enumerate}
\def\labelenumi{\arabic{enumi}.}
\tightlist
\item
  Construir una matriz 2 por 2 que contenga en filas los valores de
  Ibiza y Mallorca y por columnas las respuestas Sí y No
\item
  Con la función \texttt{prop.test} contrastar si las proporciones por
  islas son iguales o distintas.\\
\item
  Resolver el contraste con el \(p\)-valor y obtener e interpretar un
  intervalo de confianza del 95 para la diferencia de proporciones (!
  cuidado con el orden¡).
\end{enumerate}

\hypertarget{ejercicio-6}{%
\subsection{Ejercicio 6}\label{ejercicio-6}}

Se pregunta a un grupo de 100 personas elegido al azar asiste a un
webinnar sobre tecnología para la banca. Antes de la conferencia se les
pregunta si consideran que Internet es segura para la banca, después de
la conferencia se les vuelve a preguntar cual es su opinión. Los
resultados fueron los siguientes:

\begin{center}
\begin{tabular}{|c|c|cc|}
\cline{3-4}
     \multicolumn{2}{c|}{}& \multicolumn{2}{|c|} {Después}\\\cline{3-4}
   \multicolumn{2}{c|}{} & Sí Segura & No Segura \\\hline
Antes & Sí  Segura &  50 &  30 \\
    & No Segura   &  5 & 15 
\\\hline
\end{tabular}
\end{center}

\end{document}
