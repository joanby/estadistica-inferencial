% Options for packages loaded elsewhere
\PassOptionsToPackage{unicode}{hyperref}
\PassOptionsToPackage{hyphens}{url}
\PassOptionsToPackage{dvipsnames,svgnames,x11names}{xcolor}
%
\documentclass[
]{article}
\title{Problemas de Análisis de la varianza}
\author{}
\date{\vspace{-2.5em}}

\usepackage{amsmath,amssymb}
\usepackage{lmodern}
\usepackage{iftex}
\ifPDFTeX
  \usepackage[T1]{fontenc}
  \usepackage[utf8]{inputenc}
  \usepackage{textcomp} % provide euro and other symbols
\else % if luatex or xetex
  \usepackage{unicode-math}
  \defaultfontfeatures{Scale=MatchLowercase}
  \defaultfontfeatures[\rmfamily]{Ligatures=TeX,Scale=1}
\fi
% Use upquote if available, for straight quotes in verbatim environments
\IfFileExists{upquote.sty}{\usepackage{upquote}}{}
\IfFileExists{microtype.sty}{% use microtype if available
  \usepackage[]{microtype}
  \UseMicrotypeSet[protrusion]{basicmath} % disable protrusion for tt fonts
}{}
\makeatletter
\@ifundefined{KOMAClassName}{% if non-KOMA class
  \IfFileExists{parskip.sty}{%
    \usepackage{parskip}
  }{% else
    \setlength{\parindent}{0pt}
    \setlength{\parskip}{6pt plus 2pt minus 1pt}}
}{% if KOMA class
  \KOMAoptions{parskip=half}}
\makeatother
\usepackage{xcolor}
\IfFileExists{xurl.sty}{\usepackage{xurl}}{} % add URL line breaks if available
\IfFileExists{bookmark.sty}{\usepackage{bookmark}}{\usepackage{hyperref}}
\hypersetup{
  pdftitle={Problemas de Análisis de la varianza},
  colorlinks=true,
  linkcolor={red},
  filecolor={Maroon},
  citecolor={blue},
  urlcolor={blue},
  pdfcreator={LaTeX via pandoc}}
\urlstyle{same} % disable monospaced font for URLs
\usepackage[left=2cm,right=2cm,top=2cm,bottom=2cm]{geometry}
\usepackage{color}
\usepackage{fancyvrb}
\newcommand{\VerbBar}{|}
\newcommand{\VERB}{\Verb[commandchars=\\\{\}]}
\DefineVerbatimEnvironment{Highlighting}{Verbatim}{commandchars=\\\{\}}
% Add ',fontsize=\small' for more characters per line
\usepackage{framed}
\definecolor{shadecolor}{RGB}{248,248,248}
\newenvironment{Shaded}{\begin{snugshade}}{\end{snugshade}}
\newcommand{\AlertTok}[1]{\textcolor[rgb]{0.94,0.16,0.16}{#1}}
\newcommand{\AnnotationTok}[1]{\textcolor[rgb]{0.56,0.35,0.01}{\textbf{\textit{#1}}}}
\newcommand{\AttributeTok}[1]{\textcolor[rgb]{0.77,0.63,0.00}{#1}}
\newcommand{\BaseNTok}[1]{\textcolor[rgb]{0.00,0.00,0.81}{#1}}
\newcommand{\BuiltInTok}[1]{#1}
\newcommand{\CharTok}[1]{\textcolor[rgb]{0.31,0.60,0.02}{#1}}
\newcommand{\CommentTok}[1]{\textcolor[rgb]{0.56,0.35,0.01}{\textit{#1}}}
\newcommand{\CommentVarTok}[1]{\textcolor[rgb]{0.56,0.35,0.01}{\textbf{\textit{#1}}}}
\newcommand{\ConstantTok}[1]{\textcolor[rgb]{0.00,0.00,0.00}{#1}}
\newcommand{\ControlFlowTok}[1]{\textcolor[rgb]{0.13,0.29,0.53}{\textbf{#1}}}
\newcommand{\DataTypeTok}[1]{\textcolor[rgb]{0.13,0.29,0.53}{#1}}
\newcommand{\DecValTok}[1]{\textcolor[rgb]{0.00,0.00,0.81}{#1}}
\newcommand{\DocumentationTok}[1]{\textcolor[rgb]{0.56,0.35,0.01}{\textbf{\textit{#1}}}}
\newcommand{\ErrorTok}[1]{\textcolor[rgb]{0.64,0.00,0.00}{\textbf{#1}}}
\newcommand{\ExtensionTok}[1]{#1}
\newcommand{\FloatTok}[1]{\textcolor[rgb]{0.00,0.00,0.81}{#1}}
\newcommand{\FunctionTok}[1]{\textcolor[rgb]{0.00,0.00,0.00}{#1}}
\newcommand{\ImportTok}[1]{#1}
\newcommand{\InformationTok}[1]{\textcolor[rgb]{0.56,0.35,0.01}{\textbf{\textit{#1}}}}
\newcommand{\KeywordTok}[1]{\textcolor[rgb]{0.13,0.29,0.53}{\textbf{#1}}}
\newcommand{\NormalTok}[1]{#1}
\newcommand{\OperatorTok}[1]{\textcolor[rgb]{0.81,0.36,0.00}{\textbf{#1}}}
\newcommand{\OtherTok}[1]{\textcolor[rgb]{0.56,0.35,0.01}{#1}}
\newcommand{\PreprocessorTok}[1]{\textcolor[rgb]{0.56,0.35,0.01}{\textit{#1}}}
\newcommand{\RegionMarkerTok}[1]{#1}
\newcommand{\SpecialCharTok}[1]{\textcolor[rgb]{0.00,0.00,0.00}{#1}}
\newcommand{\SpecialStringTok}[1]{\textcolor[rgb]{0.31,0.60,0.02}{#1}}
\newcommand{\StringTok}[1]{\textcolor[rgb]{0.31,0.60,0.02}{#1}}
\newcommand{\VariableTok}[1]{\textcolor[rgb]{0.00,0.00,0.00}{#1}}
\newcommand{\VerbatimStringTok}[1]{\textcolor[rgb]{0.31,0.60,0.02}{#1}}
\newcommand{\WarningTok}[1]{\textcolor[rgb]{0.56,0.35,0.01}{\textbf{\textit{#1}}}}
\usepackage{graphicx}
\makeatletter
\def\maxwidth{\ifdim\Gin@nat@width>\linewidth\linewidth\else\Gin@nat@width\fi}
\def\maxheight{\ifdim\Gin@nat@height>\textheight\textheight\else\Gin@nat@height\fi}
\makeatother
% Scale images if necessary, so that they will not overflow the page
% margins by default, and it is still possible to overwrite the defaults
% using explicit options in \includegraphics[width, height, ...]{}
\setkeys{Gin}{width=\maxwidth,height=\maxheight,keepaspectratio}
% Set default figure placement to htbp
\makeatletter
\def\fps@figure{htbp}
\makeatother
\setlength{\emergencystretch}{3em} % prevent overfull lines
\providecommand{\tightlist}{%
  \setlength{\itemsep}{0pt}\setlength{\parskip}{0pt}}
\setcounter{secnumdepth}{5}
\renewcommand{\contentsname}{Contenidos}
\ifLuaTeX
  \usepackage{selnolig}  % disable illegal ligatures
\fi

\begin{document}
\maketitle

{
\hypersetup{linkcolor=blue}
\setcounter{tocdepth}{4}
\tableofcontents
}
\hypertarget{ejercicios-independencia-y-homogeneidad}{%
\section{Ejercicios independencia y
homogeneidad}\label{ejercicios-independencia-y-homogeneidad}}

\hypertarget{problema-1}{%
\subsection{Problema 1}\label{problema-1}}

Doce personas son distribuidas en \(4\) grupos de personas \(3\) cada
uno. A cada grupo, se le asigna aleatoriamente un tiempo distinto de
entrenamiento antes de realizar una tarea. Los resultados en la
mencionada tarea, con el correspondiente tiempo de entrenamiento, son
los siguientes:

\begin{center}
\begin{tabular}{|c|c|c|c|}
\hline
$0.5$ horas&$1$ hora&$1.5$ horas&$2$ horas\\\hline\hline
$1$&$4$&$3$&$\ \,8$\\\hline
$3$&$6$&$5$&$10$\\\hline
$5$&$2$&$7$&$\ \,6$\\\hline
\end{tabular}
\end{center}

Ver si podemos rechazar la hipótesis nula
\(H_0:\mu_1=\mu_2=\mu_3=\mu_4.\)

\hypertarget{soluciuxf3n}{%
\subsubsection{Solución}\label{soluciuxf3n}}

En primer lugar, tenemos que definir la tabla de datos para poder
aplicar el test ANOVA:

\begin{Shaded}
\begin{Highlighting}[]
\NormalTok{tarea}\OtherTok{=}\FunctionTok{c}\NormalTok{(}\DecValTok{1}\NormalTok{,}\DecValTok{3}\NormalTok{,}\DecValTok{5}\NormalTok{,}\DecValTok{4}\NormalTok{,}\DecValTok{6}\NormalTok{,}\DecValTok{2}\NormalTok{,}\DecValTok{3}\NormalTok{,}\DecValTok{5}\NormalTok{,}\DecValTok{7}\NormalTok{,}\DecValTok{8}\NormalTok{,}\DecValTok{10}\NormalTok{,}\DecValTok{6}\NormalTok{)}
\NormalTok{tiempo }\OtherTok{=} \FunctionTok{as.factor}\NormalTok{(}\FunctionTok{rep}\NormalTok{(}\FunctionTok{c}\NormalTok{(}\StringTok{"0.5"}\NormalTok{,}\StringTok{"1"}\NormalTok{,}\StringTok{"1.5"}\NormalTok{,}\StringTok{"2"}\NormalTok{),}\AttributeTok{each=}\DecValTok{3}\NormalTok{))}
\NormalTok{(}\AttributeTok{datos=}\FunctionTok{data.frame}\NormalTok{(tarea,tiempo))}
\end{Highlighting}
\end{Shaded}

\begin{verbatim}
##    tarea tiempo
## 1      1    0.5
## 2      3    0.5
## 3      5    0.5
## 4      4      1
## 5      6      1
## 6      2      1
## 7      3    1.5
## 8      5    1.5
## 9      7    1.5
## 10     8      2
## 11    10      2
## 12     6      2
\end{verbatim}

Una vez definida la tabla, realizamos el contraste ANOVA:

\begin{Shaded}
\begin{Highlighting}[]
\FunctionTok{summary}\NormalTok{(}\FunctionTok{aov}\NormalTok{(datos}\SpecialCharTok{$}\NormalTok{tarea }\SpecialCharTok{\textasciitilde{}}\NormalTok{ datos}\SpecialCharTok{$}\NormalTok{tiempo))}
\end{Highlighting}
\end{Shaded}

\begin{verbatim}
##              Df Sum Sq Mean Sq F value Pr(>F)  
## datos$tiempo  3     42      14     3.5 0.0695 .
## Residuals     8     32       4                 
## ---
## Signif. codes:  0 '***' 0.001 '**' 0.01 '*' 0.05 '.' 0.1 ' ' 1
\end{verbatim}

El p-valor está en la zona de penumbra, es decir, está entre 0.05 y 1.
Por tanto, no podemos tomar una decisión clara. Si ponemos como umbral
0.05, podríamos concluir que no tenemos evidencias suficientes para
rechazar que los resultados en el entrenamiento son distintos según el
tiempo usado.

Aunque no se pide comprobaremos la igualdad de varianzas

\begin{Shaded}
\begin{Highlighting}[]
\FunctionTok{bartlett.test}\NormalTok{(datos}\SpecialCharTok{$}\NormalTok{tarea }\SpecialCharTok{\textasciitilde{}}\NormalTok{ datos}\SpecialCharTok{$}\NormalTok{tiempo)}
\end{Highlighting}
\end{Shaded}

\begin{verbatim}
## 
##  Bartlett test of homogeneity of variances
## 
## data:  datos$tarea by datos$tiempo
## Bartlett's K-squared = 0, df = 3, p-value = 1
\end{verbatim}

\begin{Shaded}
\begin{Highlighting}[]
\FunctionTok{library}\NormalTok{(car)}
\end{Highlighting}
\end{Shaded}

\begin{verbatim}
## Loading required package: carData
\end{verbatim}

\begin{Shaded}
\begin{Highlighting}[]
\FunctionTok{leveneTest}\NormalTok{(datos}\SpecialCharTok{$}\NormalTok{tarea }\SpecialCharTok{\textasciitilde{}}\NormalTok{ datos}\SpecialCharTok{$}\NormalTok{tiempo)}
\end{Highlighting}
\end{Shaded}

\begin{verbatim}
## Levene's Test for Homogeneity of Variance (center = median)
##       Df F value Pr(>F)
## group  3       0      1
##        8
\end{verbatim}

Comprobemos las sumas de los cuadrados

\begin{Shaded}
\begin{Highlighting}[]
\NormalTok{ni}\OtherTok{=}\FunctionTok{c}\NormalTok{(}\DecValTok{3}\NormalTok{,}\DecValTok{3}\NormalTok{,}\DecValTok{3}\NormalTok{,}\DecValTok{3}\NormalTok{)}
\NormalTok{k}\OtherTok{=}\DecValTok{4}
\NormalTok{N}\OtherTok{=}\FunctionTok{sum}\NormalTok{(ni)}
\NormalTok{SST}\OtherTok{=} \FunctionTok{sum}\NormalTok{(datos}\SpecialCharTok{$}\NormalTok{tarea}\SpecialCharTok{\^{}}\DecValTok{2}\NormalTok{)}\SpecialCharTok{{-}} \FunctionTok{sum}\NormalTok{(datos}\SpecialCharTok{$}\NormalTok{tarea)}\SpecialCharTok{\^{}}\DecValTok{2}\SpecialCharTok{/}\NormalTok{N}
\NormalTok{SST}
\end{Highlighting}
\end{Shaded}

\begin{verbatim}
## [1] 74
\end{verbatim}

\begin{Shaded}
\begin{Highlighting}[]
\NormalTok{Sumas\_col}\OtherTok{=}\FunctionTok{aggregate}\NormalTok{(datos}\SpecialCharTok{$}\NormalTok{tarea,}\AttributeTok{by=}\FunctionTok{list}\NormalTok{(datos}\SpecialCharTok{$}\NormalTok{tiempo),sum)}
\NormalTok{Sumas\_col}\SpecialCharTok{$}\NormalTok{x}\SpecialCharTok{/}\NormalTok{ni}
\end{Highlighting}
\end{Shaded}

\begin{verbatim}
## [1] 3 4 5 8
\end{verbatim}

\begin{Shaded}
\begin{Highlighting}[]
\NormalTok{SSTr}\OtherTok{=}\FunctionTok{sum}\NormalTok{(Sumas\_col}\SpecialCharTok{$}\NormalTok{x}\SpecialCharTok{\^{}}\DecValTok{2}\SpecialCharTok{/}\NormalTok{ni)}\SpecialCharTok{{-}}\FunctionTok{sum}\NormalTok{(datos}\SpecialCharTok{$}\NormalTok{tarea)}\SpecialCharTok{\^{}}\DecValTok{2}\SpecialCharTok{/}\NormalTok{N}
\NormalTok{SSTr}
\end{Highlighting}
\end{Shaded}

\begin{verbatim}
## [1] 42
\end{verbatim}

\begin{Shaded}
\begin{Highlighting}[]
\NormalTok{SSE}\OtherTok{=}\NormalTok{SST}\SpecialCharTok{{-}}\NormalTok{SSTr}
\NormalTok{SSE}
\end{Highlighting}
\end{Shaded}

\begin{verbatim}
## [1] 32
\end{verbatim}

eL \(p\)-valor es

\begin{Shaded}
\begin{Highlighting}[]
\NormalTok{Fest}\OtherTok{=}\NormalTok{(SSTr}\SpecialCharTok{/}\DecValTok{3}\NormalTok{)}\SpecialCharTok{/}\NormalTok{(SSE}\SpecialCharTok{/}\DecValTok{8}\NormalTok{)}
\NormalTok{Fest}
\end{Highlighting}
\end{Shaded}

\begin{verbatim}
## [1] 3.5
\end{verbatim}

\begin{Shaded}
\begin{Highlighting}[]
\DecValTok{1}\SpecialCharTok{{-}}\FunctionTok{pf}\NormalTok{(Fest,}\DecValTok{3}\NormalTok{,}\DecValTok{8}\NormalTok{)}
\end{Highlighting}
\end{Shaded}

\begin{verbatim}
## [1] 0.06949856
\end{verbatim}

\begin{Shaded}
\begin{Highlighting}[]
\FunctionTok{pf}\NormalTok{(Fest,}\DecValTok{3}\NormalTok{,}\DecValTok{8}\NormalTok{,}\AttributeTok{lower.tail=}\ConstantTok{FALSE}\NormalTok{)}
\end{Highlighting}
\end{Shaded}

\begin{verbatim}
## [1] 0.06949856
\end{verbatim}

\hypertarget{problema-2}{%
\subsection{Problema 2}\label{problema-2}}

Se registraron las frecuencias de los días que llovió a diferentes
horas, durante los meses de enero, marzo, mayo y julio. Los datos
obtenidos, durante un periodo de 10 años, fueron los siguientes:
~\newline

\begin{center}
\begin{tabular}{|c||c|c|c|c||c|}
\hline
Hora&enero&febrero&marzo&julio&Total\\\hline\hline
$\ \,9$&$\ \,22$&$\ \,25$&$\ \,24$&$\ \,11$&$\ \,82$\\\hline
$10$&$\ \,21$&$\ \,19$&$\ \,18$&$\ \,16$&$\ \,74$\\\hline
$11$&$\ \,17$&$\ \,23$&$\ \,26$&$\ \,17$&$\ \,83$\\\hline
$12$&$\ \,20$&$\ \,31$&$\ \,25$&$\ \,24$&$100$\\\hline
$13$&$\ \,16$&$\ \,15$&$\ \,23$&$\ \,24$&$\ \,78$\\\hline
$14$&$\ \,21$&$\ \,35$&$\ \,23$&$\ \,20$&$\ \,99$\\\hline\hline
Total&$117$&$148$&$139$&$112$&$536$\\\hline
\end{tabular}
\end{center}

~\newline Estudiar la variabilidad entre meses y entre horas.

\hypertarget{soluciuxf3n-1}{%
\subsubsection{Solución}\label{soluciuxf3n-1}}

En primer lugar, tenemos que definir la tabla de datos para poder
aplicar el test ANOVA:

\begin{Shaded}
\begin{Highlighting}[]
\NormalTok{frecuencias }\OtherTok{=} \FunctionTok{c}\NormalTok{(}\DecValTok{22}\NormalTok{,}\DecValTok{25}\NormalTok{,}\DecValTok{24}\NormalTok{,}\DecValTok{11}\NormalTok{,}\DecValTok{21}\NormalTok{,}\DecValTok{19}\NormalTok{,}\DecValTok{18}\NormalTok{,}\DecValTok{16}\NormalTok{,}\DecValTok{17}\NormalTok{,}\DecValTok{23}\NormalTok{,}\DecValTok{26}\NormalTok{,}\DecValTok{17}\NormalTok{,}\DecValTok{20}\NormalTok{,}\DecValTok{31}\NormalTok{,}\DecValTok{25}\NormalTok{,}\DecValTok{24}\NormalTok{,}\DecValTok{16}\NormalTok{,}\DecValTok{15}\NormalTok{,}\DecValTok{23}\NormalTok{,}\DecValTok{24}\NormalTok{,}\DecValTok{21}\NormalTok{,}\DecValTok{35}\NormalTok{,}\DecValTok{23}\NormalTok{,}\DecValTok{20}\NormalTok{)}
\NormalTok{horas }\OtherTok{=} \FunctionTok{as.factor}\NormalTok{(}\FunctionTok{rep}\NormalTok{(}\FunctionTok{c}\NormalTok{(}\StringTok{"9"}\NormalTok{,}\StringTok{"10"}\NormalTok{,}\StringTok{"11"}\NormalTok{,}\StringTok{"12"}\NormalTok{,}\StringTok{"13"}\NormalTok{,}\StringTok{"14"}\NormalTok{),}\AttributeTok{each=}\DecValTok{4}\NormalTok{))}
\NormalTok{meses }\OtherTok{=} \FunctionTok{as.factor}\NormalTok{(}\FunctionTok{rep}\NormalTok{(}\FunctionTok{c}\NormalTok{(}\StringTok{"enero"}\NormalTok{,}\StringTok{"febrero"}\NormalTok{,}\StringTok{"marzo"}\NormalTok{,}\StringTok{"julio"}\NormalTok{),}\DecValTok{6}\NormalTok{))}
\NormalTok{(}\AttributeTok{datos =} \FunctionTok{data.frame}\NormalTok{(horas,meses,frecuencias))}
\end{Highlighting}
\end{Shaded}

\begin{verbatim}
##    horas   meses frecuencias
## 1      9   enero          22
## 2      9 febrero          25
## 3      9   marzo          24
## 4      9   julio          11
## 5     10   enero          21
## 6     10 febrero          19
## 7     10   marzo          18
## 8     10   julio          16
## 9     11   enero          17
## 10    11 febrero          23
## 11    11   marzo          26
## 12    11   julio          17
## 13    12   enero          20
## 14    12 febrero          31
## 15    12   marzo          25
## 16    12   julio          24
## 17    13   enero          16
## 18    13 febrero          15
## 19    13   marzo          23
## 20    13   julio          24
## 21    14   enero          21
## 22    14 febrero          35
## 23    14   marzo          23
## 24    14   julio          20
\end{verbatim}

Una vez definida la tabla, realizamos el contraste ANOVA:

\begin{Shaded}
\begin{Highlighting}[]
\FunctionTok{summary}\NormalTok{(}\FunctionTok{aov}\NormalTok{(datos}\SpecialCharTok{$}\NormalTok{frecuencias }\SpecialCharTok{\textasciitilde{}}\NormalTok{ datos}\SpecialCharTok{$}\NormalTok{horas }\SpecialCharTok{+}\NormalTok{ datos}\SpecialCharTok{$}\NormalTok{meses))}
\end{Highlighting}
\end{Shaded}

\begin{verbatim}
##             Df Sum Sq Mean Sq F value Pr(>F)
## datos$horas  5  149.5   29.90   1.395  0.282
## datos$meses  3  149.0   49.67   2.317  0.117
## Residuals   15  321.5   21.43
\end{verbatim}

Como los p-valores por horas y por meses son grandes, concluimos que no
tenemos evidencias para rechazar que el número de días que llueve por
mes no depende ni del mes ni de la hora del día en que llueve.

\hypertarget{problema-3}{%
\subsection{Problema 3}\label{problema-3}}

Se realizó un estudio para determinar el nivel de agua y el tipo de
planta sobre la longitud global del tronco de las plantas de guisantes.
Se utilizaron \(3\) niveles de agua y \(2\) tipos de plantas. Se dispone
para el estudio de \(18\) plantas sin hojas. Las plantas se dividen
aleatoriamente en \(3\) subgrupos y después se los asigna los niveles de
agua aleatoriamente. Se sigue un procedimiento parecido con \(18\)
plantas convencionales. Se obtuvieron los resultados siguientes (la
longitud del tronco se da en centímetros): ~\newline

\begin{center}
\begin{tabular}{c|c|c|c|c|}
&&\multicolumn{3}{c|}{FACTOR AGUA}\\\hline
& &{bajo}&{medio}&{alto}\\\cline{3-5}
\multirow{12}{1.75cm}{FACTOR PLANTA}&\multirow{6}{1cm}{Sin Hojas}&
$69.0$&$\ \,96.1$&$121.0$\\\cline{3-5}
&&$71.3$&$102.3$&$122.9$\\\cline{3-5}
&&$73.2$&$107.5$&$123.1$\\\cline{3-5}
&&$75.1$&$103.6$&$125.7$\\\cline{3-5}
&&$74.4$&$100.7$&$125.2$\\\cline{3-5}
&&$75.0$&$101.8$&$120.1$\\\cline{2-5}
&\multirow{6}{1cm}{Con Hojas}&$71.1$&$\ \,81.0$&$101.1$\\\cline{3-5}
&&$69.2$&$\ \,85.8$&$103.2$\\\cline{3-5}
&&$70.4$&$\ \,86.0$&$106.1$\\\cline{3-5}
&&$73.2$&$\ \,87.5$&$109.7$\\\cline{3-5}
&&$71.2$&$\ \,88.1$&$109.0$\\\cline{3-5}
&&$70.9$&$\ \,87.6$&$106.9$\\\hline
\end{tabular}
\end{center}

~\newline Se desea saber si hay diferencias entre los niveles de agua y
entre los diferentes tipos de planta. También se quiere saber si hay
interacción entre los niveles de agua y el tipo de planta.

\hypertarget{soluciuxf3n-2}{%
\subsubsection{Solución}\label{soluciuxf3n-2}}

En primer lugar, tenemos que definir la tabla de datos para poder
aplicar el test ANOVA:

\begin{Shaded}
\begin{Highlighting}[]
\NormalTok{longitud }\OtherTok{=} \FunctionTok{c}\NormalTok{(}\DecValTok{69}\NormalTok{,}\FloatTok{96.1}\NormalTok{,}\DecValTok{121}\NormalTok{,}\FloatTok{71.3}\NormalTok{,}\FloatTok{102.3}\NormalTok{,}\FloatTok{122.9}\NormalTok{,}\FloatTok{73.2}\NormalTok{,}\FloatTok{107.5}\NormalTok{,}\FloatTok{123.1}\NormalTok{,}\FloatTok{75.1}\NormalTok{,}\FloatTok{103.6}\NormalTok{,}\FloatTok{125.7}\NormalTok{,}\FloatTok{74.4}\NormalTok{,}\FloatTok{100.7}\NormalTok{,}\FloatTok{125.2}\NormalTok{,}
             \DecValTok{75}\NormalTok{,}\FloatTok{101.8}\NormalTok{,}\FloatTok{120.1}\NormalTok{,}\FloatTok{71.1}\NormalTok{,}\DecValTok{81}\NormalTok{,}\FloatTok{101.1}\NormalTok{,}\FloatTok{69.2}\NormalTok{,}\FloatTok{85.8}\NormalTok{,}\FloatTok{103.2}\NormalTok{,}\FloatTok{70.4}\NormalTok{,}\DecValTok{86}\NormalTok{,}\FloatTok{106.1}\NormalTok{,}\FloatTok{73.2}\NormalTok{,}\FloatTok{87.5}\NormalTok{,}\FloatTok{109.7}\NormalTok{,}
             \FloatTok{71.2}\NormalTok{,}\FloatTok{88.1}\NormalTok{,}\DecValTok{109}\NormalTok{,}\FloatTok{70.9}\NormalTok{,}\FloatTok{87.6}\NormalTok{,}\FloatTok{106.9}\NormalTok{)}
\NormalTok{factor.agua }\OtherTok{=} \FunctionTok{as.factor}\NormalTok{(}\FunctionTok{rep}\NormalTok{(}\FunctionTok{c}\NormalTok{(}\StringTok{"bajo"}\NormalTok{,}\StringTok{"medio"}\NormalTok{,}\StringTok{"alto"}\NormalTok{),}\DecValTok{12}\NormalTok{))}
\NormalTok{factor.planta }\OtherTok{=} \FunctionTok{as.factor}\NormalTok{(}\FunctionTok{rep}\NormalTok{(}\FunctionTok{c}\NormalTok{(}\StringTok{"sin hojas"}\NormalTok{,}\StringTok{"con hojas"}\NormalTok{),}\AttributeTok{each=}\DecValTok{18}\NormalTok{))}
\NormalTok{(}\AttributeTok{datos=}\FunctionTok{data.frame}\NormalTok{(factor.agua,factor.planta,longitud))}
\end{Highlighting}
\end{Shaded}

\begin{verbatim}
##    factor.agua factor.planta longitud
## 1         bajo     sin hojas     69.0
## 2        medio     sin hojas     96.1
## 3         alto     sin hojas    121.0
## 4         bajo     sin hojas     71.3
## 5        medio     sin hojas    102.3
## 6         alto     sin hojas    122.9
## 7         bajo     sin hojas     73.2
## 8        medio     sin hojas    107.5
## 9         alto     sin hojas    123.1
## 10        bajo     sin hojas     75.1
## 11       medio     sin hojas    103.6
## 12        alto     sin hojas    125.7
## 13        bajo     sin hojas     74.4
## 14       medio     sin hojas    100.7
## 15        alto     sin hojas    125.2
## 16        bajo     sin hojas     75.0
## 17       medio     sin hojas    101.8
## 18        alto     sin hojas    120.1
## 19        bajo     con hojas     71.1
## 20       medio     con hojas     81.0
## 21        alto     con hojas    101.1
## 22        bajo     con hojas     69.2
## 23       medio     con hojas     85.8
## 24        alto     con hojas    103.2
## 25        bajo     con hojas     70.4
## 26       medio     con hojas     86.0
## 27        alto     con hojas    106.1
## 28        bajo     con hojas     73.2
## 29       medio     con hojas     87.5
## 30        alto     con hojas    109.7
## 31        bajo     con hojas     71.2
## 32       medio     con hojas     88.1
## 33        alto     con hojas    109.0
## 34        bajo     con hojas     70.9
## 35       medio     con hojas     87.6
## 36        alto     con hojas    106.9
\end{verbatim}

Una vez definida la tabla, realizamos el contraste ANOVA:

\begin{Shaded}
\begin{Highlighting}[]
\FunctionTok{summary}\NormalTok{(}\FunctionTok{aov}\NormalTok{(datos}\SpecialCharTok{$}\NormalTok{longitud }\SpecialCharTok{\textasciitilde{}}\NormalTok{ datos}\SpecialCharTok{$}\NormalTok{factor.agua }\SpecialCharTok{*}\NormalTok{ datos}\SpecialCharTok{$}\NormalTok{factor.planta))}
\end{Highlighting}
\end{Shaded}

\begin{verbatim}
##                                       Df Sum Sq Mean Sq F value   Pr(>F)    
## datos$factor.agua                      2  10842    5421  734.49  < 2e-16 ***
## datos$factor.planta                    1   1225    1225  165.97 9.27e-14 ***
## datos$factor.agua:datos$factor.planta  2    422     211   28.59 1.12e-07 ***
## Residuals                             30    221       7                     
## ---
## Signif. codes:  0 '***' 0.001 '**' 0.01 '*' 0.05 '.' 0.1 ' ' 1
\end{verbatim}

Como todos los p-valores son pequeños, concluimos lo siguiente:

\begin{itemize}
\tightlist
\item
  tenemos evidencias suficientes para afirmar que la longitud de la
  planta depende del nivel de agua,
\item
  tenemos evidencias suficientes para afirmar que la longitud de la
  planta depende del tipo de planta, es decir, si ésta es sin hojas o
  con hojas y,
\item
  tenemos evidencias suficientes para afirmar que existe interacción
  entre el nivel de agua y el tipo de planta. Realicemos un gráfico de
  la interacción para comprobar gráficamente dicha evidencia:
\end{itemize}

\begin{Shaded}
\begin{Highlighting}[]
\FunctionTok{interaction.plot}\NormalTok{(datos}\SpecialCharTok{$}\NormalTok{factor.agua,datos}\SpecialCharTok{$}\NormalTok{factor.planta,datos}\SpecialCharTok{$}\NormalTok{longitud)}
\end{Highlighting}
\end{Shaded}

\includegraphics{tema7_ANOVA_SOLUCIONES_files/figure-latex/unnamed-chunk-11-1.pdf}

Observamos que los segmentos anteriores están lejos de ser paralelos.

\hypertarget{problema-4}{%
\subsection{Problema 4}\label{problema-4}}

Las variables aleatorias \(X_i\) siguen la distribución
\(N(m_i,\sigma^2),\ i=1,2,3,4\). Consideramos las siguientes muestras de
tamaños \(n_i=7\) de las mencionadas variables aleatorias: ~\newline

\begin{center}
\begin{tabular}{cccccccc}
$X_1$&$20$&$26$&$26$&$24$&$23$&$26$&$21$\\
$X_2$&$24$&$22$&$20$&$21$&$21$&$22$&$20$\\
$X_3$&$16$&$18$&$20$&$21$&$24$&$15$&$17$\\
$X_4$&$19$&$15$&$13$&$16$&$12$&$11$&$14$\\
\end{tabular}
\end{center}

~\newline a) Comprobar si las varianzas son iguales. b) Contrastar la
igualdad de medias.

\hypertarget{soluciuxf3n-3}{%
\subsubsection{Solución}\label{soluciuxf3n-3}}

En primer lugar, tenemos que definir la tabla de datos para poder
aplicar el test ANOVA:

\begin{Shaded}
\begin{Highlighting}[]
\NormalTok{valores}\OtherTok{=}\FunctionTok{c}\NormalTok{(}\DecValTok{20}\NormalTok{,}\DecValTok{26}\NormalTok{,}\DecValTok{26}\NormalTok{,}\DecValTok{24}\NormalTok{,}\DecValTok{23}\NormalTok{,}\DecValTok{26}\NormalTok{,}\DecValTok{21}\NormalTok{,}\DecValTok{24}\NormalTok{,}\DecValTok{22}\NormalTok{,}\DecValTok{20}\NormalTok{,}\DecValTok{21}\NormalTok{,}\DecValTok{21}\NormalTok{,}\DecValTok{22}\NormalTok{,}\DecValTok{20}\NormalTok{,}\DecValTok{16}\NormalTok{,}\DecValTok{18}\NormalTok{,}\DecValTok{20}\NormalTok{,}
          \DecValTok{21}\NormalTok{,}\DecValTok{24}\NormalTok{,}\DecValTok{15}\NormalTok{,}\DecValTok{17}\NormalTok{,}\DecValTok{19}\NormalTok{,}\DecValTok{15}\NormalTok{,}\DecValTok{13}\NormalTok{,}\DecValTok{16}\NormalTok{,}\DecValTok{12}\NormalTok{,}\DecValTok{11}\NormalTok{,}\DecValTok{14}\NormalTok{)}
\NormalTok{variable.aleatoria }\OtherTok{=} \FunctionTok{as.factor}\NormalTok{(}\FunctionTok{rep}\NormalTok{(}\FunctionTok{c}\NormalTok{(}\StringTok{"X1"}\NormalTok{,}\StringTok{"X2"}\NormalTok{,}\StringTok{"X3"}\NormalTok{,}\StringTok{"X4"}\NormalTok{),}\AttributeTok{each=}\DecValTok{7}\NormalTok{))}
\NormalTok{(}\AttributeTok{datos=}\FunctionTok{data.frame}\NormalTok{(valores,variable.aleatoria))}
\end{Highlighting}
\end{Shaded}

\begin{verbatim}
##    valores variable.aleatoria
## 1       20                 X1
## 2       26                 X1
## 3       26                 X1
## 4       24                 X1
## 5       23                 X1
## 6       26                 X1
## 7       21                 X1
## 8       24                 X2
## 9       22                 X2
## 10      20                 X2
## 11      21                 X2
## 12      21                 X2
## 13      22                 X2
## 14      20                 X2
## 15      16                 X3
## 16      18                 X3
## 17      20                 X3
## 18      21                 X3
## 19      24                 X3
## 20      15                 X3
## 21      17                 X3
## 22      19                 X4
## 23      15                 X4
## 24      13                 X4
## 25      16                 X4
## 26      12                 X4
## 27      11                 X4
## 28      14                 X4
\end{verbatim}

Para contrastar si las varianzas son iguales, usamos el test de
Bartlett:

\begin{Shaded}
\begin{Highlighting}[]
\FunctionTok{bartlett.test}\NormalTok{(valores }\SpecialCharTok{\textasciitilde{}}\NormalTok{ variable.aleatoria)}
\end{Highlighting}
\end{Shaded}

\begin{verbatim}
## 
##  Bartlett test of homogeneity of variances
## 
## data:  valores by variable.aleatoria
## Bartlett's K-squared = 3.4291, df = 3, p-value = 0.3301
\end{verbatim}

Como el p-valor es grande, concluimos que no tenemos evidencias
suficientes para rechazar que las varianzas de las muestras de las 4
variables aleatorias no sean iguales.

Contrastemos a continuación si las medias son iguales usando el test
ANOVA:

\begin{Shaded}
\begin{Highlighting}[]
\FunctionTok{summary}\NormalTok{(}\FunctionTok{aov}\NormalTok{(valores }\SpecialCharTok{\textasciitilde{}}\NormalTok{ variable.aleatoria))}
\end{Highlighting}
\end{Shaded}

\begin{verbatim}
##                    Df Sum Sq Mean Sq F value   Pr(>F)    
## variable.aleatoria  3    345  114.99   18.16 2.29e-06 ***
## Residuals          24    152    6.33                     
## ---
## Signif. codes:  0 '***' 0.001 '**' 0.01 '*' 0.05 '.' 0.1 ' ' 1
\end{verbatim}

Como el p-valor es muy pequeño concluimos que tenemos evidencias
suficientes para afirmar que las medias de las 4 variables aleatorias no
son iguales.

Comprobemos las sumas de cuadrados del ANOVA

\begin{Shaded}
\begin{Highlighting}[]
\FunctionTok{summary}\NormalTok{(}\FunctionTok{aov}\NormalTok{(valores }\SpecialCharTok{\textasciitilde{}}\NormalTok{ variable.aleatoria))}\OtherTok{{-}\textgreater{}}\NormalTok{sol\_aov}
\NormalTok{ni}\OtherTok{=}\FunctionTok{c}\NormalTok{(}\DecValTok{7}\NormalTok{,}\DecValTok{7}\NormalTok{,}\DecValTok{7}\NormalTok{,}\DecValTok{7}\NormalTok{)}
\NormalTok{k}\OtherTok{=}\DecValTok{4}
\NormalTok{N}\OtherTok{=}\FunctionTok{sum}\NormalTok{(ni)}
\NormalTok{SST}\OtherTok{=} \FunctionTok{sum}\NormalTok{(valores}\SpecialCharTok{\^{}}\DecValTok{2}\NormalTok{)}\SpecialCharTok{{-}} \FunctionTok{sum}\NormalTok{(valores)}\SpecialCharTok{\^{}}\DecValTok{2}\SpecialCharTok{/}\NormalTok{N}
\NormalTok{SST}
\end{Highlighting}
\end{Shaded}

\begin{verbatim}
## [1] 496.9643
\end{verbatim}

\begin{Shaded}
\begin{Highlighting}[]
\NormalTok{Sumas\_col}\OtherTok{=}\FunctionTok{aggregate}\NormalTok{(valores,}\AttributeTok{by=}\FunctionTok{list}\NormalTok{(variable.aleatoria),sum)}
\NormalTok{Sumas\_col}\SpecialCharTok{$}\NormalTok{x}\SpecialCharTok{/}\NormalTok{ni}
\end{Highlighting}
\end{Shaded}

\begin{verbatim}
## [1] 23.71429 21.42857 18.71429 14.28571
\end{verbatim}

\begin{Shaded}
\begin{Highlighting}[]
\NormalTok{SSTr}\OtherTok{=}\FunctionTok{sum}\NormalTok{(Sumas\_col}\SpecialCharTok{$}\NormalTok{x}\SpecialCharTok{\^{}}\DecValTok{2}\SpecialCharTok{/}\NormalTok{ni)}\SpecialCharTok{{-}}\FunctionTok{sum}\NormalTok{(valores)}\SpecialCharTok{\^{}}\DecValTok{2}\SpecialCharTok{/}\NormalTok{N}
\NormalTok{SSTr}
\end{Highlighting}
\end{Shaded}

\begin{verbatim}
## [1] 344.9643
\end{verbatim}

\begin{Shaded}
\begin{Highlighting}[]
\NormalTok{SSE}\OtherTok{=}\NormalTok{SST}\SpecialCharTok{{-}}\NormalTok{SSTr}
\NormalTok{SSE}
\end{Highlighting}
\end{Shaded}

\begin{verbatim}
## [1] 152
\end{verbatim}

Comparamos con los resultados del summary

\begin{Shaded}
\begin{Highlighting}[]
\FunctionTok{summary}\NormalTok{(}\FunctionTok{aov}\NormalTok{(valores }\SpecialCharTok{\textasciitilde{}}\NormalTok{ variable.aleatoria))}
\end{Highlighting}
\end{Shaded}

\begin{verbatim}
##                    Df Sum Sq Mean Sq F value   Pr(>F)    
## variable.aleatoria  3    345  114.99   18.16 2.29e-06 ***
## Residuals          24    152    6.33                     
## ---
## Signif. codes:  0 '***' 0.001 '**' 0.01 '*' 0.05 '.' 0.1 ' ' 1
\end{verbatim}

\begin{Shaded}
\begin{Highlighting}[]
\FunctionTok{pairwise.t.test}\NormalTok{(valores,variable.aleatoria,}\AttributeTok{p.adjust.method =} \StringTok{"none"}\NormalTok{)}
\end{Highlighting}
\end{Shaded}

\begin{verbatim}
## 
##  Pairwise comparisons using t tests with pooled SD 
## 
## data:  valores and variable.aleatoria 
## 
##    X1      X2      X3    
## X2 0.1022  -       -     
## X3 0.0011  0.0549  -     
## X4 3.0e-07 1.9e-05 0.0031
## 
## P value adjustment method: none
\end{verbatim}

\begin{Shaded}
\begin{Highlighting}[]
\FunctionTok{pairwise.t.test}\NormalTok{(valores,variable.aleatoria,}\AttributeTok{p.adjust.method =} \StringTok{"bonferroni"}\NormalTok{ )}
\end{Highlighting}
\end{Shaded}

\begin{verbatim}
## 
##  Pairwise comparisons using t tests with pooled SD 
## 
## data:  valores and variable.aleatoria 
## 
##    X1      X2      X3     
## X2 0.61328 -       -      
## X3 0.00644 0.32957 -      
## X4 1.8e-06 0.00011 0.01842
## 
## P value adjustment method: bonferroni
\end{verbatim}

\begin{Shaded}
\begin{Highlighting}[]
\FunctionTok{pairwise.t.test}\NormalTok{(valores,variable.aleatoria,}\AttributeTok{p.adjust.method =} \StringTok{"holm"}\NormalTok{ )}
\end{Highlighting}
\end{Shaded}

\begin{verbatim}
## 
##  Pairwise comparisons using t tests with pooled SD 
## 
## data:  valores and variable.aleatoria 
## 
##    X1      X2      X3    
## X2 0.1099  -       -     
## X3 0.0043  0.1099  -     
## X4 1.8e-06 9.5e-05 0.0092
## 
## P value adjustment method: holm
\end{verbatim}

\begin{Shaded}
\begin{Highlighting}[]
\FunctionTok{library}\NormalTok{(agricolae)}
\end{Highlighting}
\end{Shaded}

\begin{verbatim}
## Warning: package 'agricolae' was built under R version 4.1.3
\end{verbatim}

\begin{Shaded}
\begin{Highlighting}[]
\NormalTok{resultado.anova}\OtherTok{=}\FunctionTok{aov}\NormalTok{(valores}\SpecialCharTok{\textasciitilde{}}\NormalTok{variable.aleatoria)}
\FunctionTok{duncan.test}\NormalTok{(resultado.anova,}\StringTok{"variable.aleatoria"}\NormalTok{,}\AttributeTok{group=}\ConstantTok{TRUE}\NormalTok{,}\AttributeTok{alpha =} \FloatTok{0.05}\NormalTok{)}\SpecialCharTok{$}\NormalTok{group}
\end{Highlighting}
\end{Shaded}

\begin{verbatim}
##     valores groups
## X1 23.71429      a
## X2 21.42857     ab
## X3 18.71429      b
## X4 14.28571      c
\end{verbatim}

\end{document}
