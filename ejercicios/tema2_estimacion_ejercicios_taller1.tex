\PassOptionsToPackage{unicode=true}{hyperref} % options for packages loaded elsewhere
\PassOptionsToPackage{hyphens}{url}
\PassOptionsToPackage{dvipsnames,svgnames*,x11names*}{xcolor}
%
\documentclass[]{article}
\usepackage{lmodern}
\usepackage{amssymb,amsmath}
\usepackage{ifxetex,ifluatex}
\usepackage{fixltx2e} % provides \textsubscript
\ifnum 0\ifxetex 1\fi\ifluatex 1\fi=0 % if pdftex
  \usepackage[T1]{fontenc}
  \usepackage[utf8]{inputenc}
  \usepackage{textcomp} % provides euro and other symbols
\else % if luatex or xelatex
  \usepackage{unicode-math}
  \defaultfontfeatures{Ligatures=TeX,Scale=MatchLowercase}
\fi
% use upquote if available, for straight quotes in verbatim environments
\IfFileExists{upquote.sty}{\usepackage{upquote}}{}
% use microtype if available
\IfFileExists{microtype.sty}{%
\usepackage[]{microtype}
\UseMicrotypeSet[protrusion]{basicmath} % disable protrusion for tt fonts
}{}
\IfFileExists{parskip.sty}{%
\usepackage{parskip}
}{% else
\setlength{\parindent}{0pt}
\setlength{\parskip}{6pt plus 2pt minus 1pt}
}
\usepackage{xcolor}
\usepackage{hyperref}
\hypersetup{
            pdftitle={Ejercicios Tema 2 - Estimación. Taller 1},
            pdfauthor={Ricardo Alberich, Juan Gabriel Gomila y Arnau Mir},
            colorlinks=true,
            linkcolor=red,
            filecolor=Maroon,
            citecolor=blue,
            urlcolor=blue,
            breaklinks=true}
\urlstyle{same}  % don't use monospace font for urls
\usepackage[margin=1in]{geometry}
\usepackage{graphicx,grffile}
\makeatletter
\def\maxwidth{\ifdim\Gin@nat@width>\linewidth\linewidth\else\Gin@nat@width\fi}
\def\maxheight{\ifdim\Gin@nat@height>\textheight\textheight\else\Gin@nat@height\fi}
\makeatother
% Scale images if necessary, so that they will not overflow the page
% margins by default, and it is still possible to overwrite the defaults
% using explicit options in \includegraphics[width, height, ...]{}
\setkeys{Gin}{width=\maxwidth,height=\maxheight,keepaspectratio}
\setlength{\emergencystretch}{3em}  % prevent overfull lines
\providecommand{\tightlist}{%
  \setlength{\itemsep}{0pt}\setlength{\parskip}{0pt}}
\setcounter{secnumdepth}{5}
% Redefines (sub)paragraphs to behave more like sections
\ifx\paragraph\undefined\else
\let\oldparagraph\paragraph
\renewcommand{\paragraph}[1]{\oldparagraph{#1}\mbox{}}
\fi
\ifx\subparagraph\undefined\else
\let\oldsubparagraph\subparagraph
\renewcommand{\subparagraph}[1]{\oldsubparagraph{#1}\mbox{}}
\fi

% set default figure placement to htbp
\makeatletter
\def\fps@figure{htbp}
\makeatother

\renewcommand{\contentsname}{Contenidos}

\title{Ejercicios Tema 2 - Estimación. Taller 1}
\author{Ricardo Alberich, Juan Gabriel Gomila y Arnau Mir}
\date{Curso completo de estadística inferencial con R y Python}

\begin{document}
\maketitle

{
\hypersetup{linkcolor=blue}
\setcounter{tocdepth}{2}
\tableofcontents
}
\hypertarget{estimaciuxf3n-taller-1}{%
\section{Estimación taller 1}\label{estimaciuxf3n-taller-1}}

\hypertarget{ejercicio-1}{%
\subsection{Ejercicio 1}\label{ejercicio-1}}

El fabricante SMART\_LED fabrica bombillas led inteligentes y de alta
gama. Supongamos que la vida de de estas bombillas sigue una
distribución exponencial de parámetro \(\lambda\). Si tomamos una
muestra aleatoria de tamaño \(n\) de estas bombillas y representamos por
\(X_i\) la duración de la \(i-\)ésima bombilla para \(i=1,\ldots,n\),
¿cuál es la función de densidad conjunta de la muestra?

\hypertarget{ejercicio-2}{%
\subsection{Ejercicio 2}\label{ejercicio-2}}

Sean \(X_1,X_2,\ldots,X_{10}\) variables aleatorias que son una muestra
aleatoria simple de una v.a. \(X\). a. Dividimos la muestra en dos
partes: de forma que la primera son los \(5\) primeros valores y la
segunda los restantes. ¿Son independientes las dos partes? b. Volvemos a
dividir la muestra en dos partes: la primera está formada por los \(5\)
valores más pequeños y la segunda por el resto. ¿Son independientes las
dos partes?

\hypertarget{ejercicio-3}{%
\subsection{Ejercicio 3}\label{ejercicio-3}}

Un fabricante de motores pone a prueba \(6\) motores sobre el mismo
prototipo de coche de competición. Para probar que los motores tienes
las mismas prestaciones se someten a distintas pruebas en un circuito.
Las velocidades máximas en 10 vueltas al circuito de cada motor tras la
prueba son \(190,195,193,177,201\) y \(187\) en Km/h. Estos valores
forman una muestra aleatoria simple de la variable
\(X=\mbox{velocidad máxima de un motor en 10 vueltas.}\) Se pide
calcular los valores observados de los siguientes estadísticos de la
muestra: a. \(\overline{X}\). b. \(\tilde{S}^2\). c. Mediana. d.
\(X_{(4)}\) (valor que ocupa el cuarto lugar ordenados los valores de
menor a mayor).

\hypertarget{ejercicio-4}{%
\subsection{Ejercicio 4}\label{ejercicio-4}}

¿Cuál es la probabilidad de que el máximo de de una muestra de tamaño
\(n=10\) de una v.a. uniforme en el intervalo \((0,1)\) sea mayor que
\(0.9\)? ¿Cuál es la probabilidad de sea menor que \(\frac12\)?

\hypertarget{ejercicio-5}{%
\subsection{Ejercicio 5}\label{ejercicio-5}}

Sea \(X_1,X_2,\ldots,X_n\) una muestra aleatoria simple de una variable
aleatoria normal de parámetros \(\mu\) y \(\sigma\). Denotemos por
\(X_{(1)}\leq X_{(2)}\leq ,\ldots,\leq X_{(n)}\) la muestra ordenada de
menor a mayor. a. Calcular la funciones de densidad del mínimo
\(X_{(1)}\) y del máximo \(X_{(n)}\) b. ¿Alguna de estas variables sigue
una distribución normal?

\hypertarget{ejercicio-6}{%
\subsection{Ejercicio 6}\label{ejercicio-6}}

Consideremos la muestra aleatoria simple \(X_1,X_2,\ldots,X_n\) de una
v.a \(X\) de media \(\mu\) y varianza \(\sigma^2\) desconocidas.
Definimos
\[\overline{X}=\frac1n \sum\limits_{i=1}^n X_i\mbox{ y } T=\frac{\sqrt{n}\cdot(\overline{X}-\mu)}{\sigma}.\]

\begin{enumerate}
\def\labelenumi{\alph{enumi}.}
\tightlist
\item
  ¿Cuál es la distribución de \(T\)?
\item
  ¿Es \(T\) un estadístico?
\end{enumerate}

\hypertarget{ejercicio-7}{%
\subsection{Ejercicio 7}\label{ejercicio-7}}

Consideremos la muestra aleatoria simple \(X_1,X_2,\ldots,X_n\) de
tamaño \(n=10\) de una v.a \(X\) normal estándar. Calculad
\(P\left(2.56<\sum\limits_{i=1}^{10} X_i^2 <18.31\right)\).

\hypertarget{ejercicio-8}{%
\subsection{Ejercicio 8}\label{ejercicio-8}}

Consideremos la muestra aleatoria simple \(X_1,X_2,\ldots,X_{n}\) de
tamaño \(n=10\) de una v.a \(X\) normal \(N(\mu=2,\sigma=4)\). Definimos
la siguiente variable aleatoria
\(Y=\frac{\sum\limits_{i=1}^{10}{(X_i-2)}^2}{16}\). Calculad
\(P(Y\leq 2.6)\)

\end{document}
