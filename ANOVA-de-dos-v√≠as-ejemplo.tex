% Options for packages loaded elsewhere
\PassOptionsToPackage{unicode}{hyperref}
\PassOptionsToPackage{hyphens}{url}
%
\documentclass[
]{article}
\usepackage{lmodern}
\usepackage{amssymb,amsmath}
\usepackage{ifxetex,ifluatex}
\ifnum 0\ifxetex 1\fi\ifluatex 1\fi=0 % if pdftex
  \usepackage[T1]{fontenc}
  \usepackage[utf8]{inputenc}
  \usepackage{textcomp} % provide euro and other symbols
\else % if luatex or xetex
  \usepackage{unicode-math}
  \defaultfontfeatures{Scale=MatchLowercase}
  \defaultfontfeatures[\rmfamily]{Ligatures=TeX,Scale=1}
\fi
% Use upquote if available, for straight quotes in verbatim environments
\IfFileExists{upquote.sty}{\usepackage{upquote}}{}
\IfFileExists{microtype.sty}{% use microtype if available
  \usepackage[]{microtype}
  \UseMicrotypeSet[protrusion]{basicmath} % disable protrusion for tt fonts
}{}
\makeatletter
\@ifundefined{KOMAClassName}{% if non-KOMA class
  \IfFileExists{parskip.sty}{%
    \usepackage{parskip}
  }{% else
    \setlength{\parindent}{0pt}
    \setlength{\parskip}{6pt plus 2pt minus 1pt}}
}{% if KOMA class
  \KOMAoptions{parskip=half}}
\makeatother
\usepackage{xcolor}
\IfFileExists{xurl.sty}{\usepackage{xurl}}{} % add URL line breaks if available
\IfFileExists{bookmark.sty}{\usepackage{bookmark}}{\usepackage{hyperref}}
\hypersetup{
  pdftitle={Ejemplo anova de dos vías},
  pdfauthor={Luis Gerardo Guzmán Rojas},
  hidelinks,
  pdfcreator={LaTeX via pandoc}}
\urlstyle{same} % disable monospaced font for URLs
\usepackage[margin=1in]{geometry}
\usepackage{color}
\usepackage{fancyvrb}
\newcommand{\VerbBar}{|}
\newcommand{\VERB}{\Verb[commandchars=\\\{\}]}
\DefineVerbatimEnvironment{Highlighting}{Verbatim}{commandchars=\\\{\}}
% Add ',fontsize=\small' for more characters per line
\usepackage{framed}
\definecolor{shadecolor}{RGB}{248,248,248}
\newenvironment{Shaded}{\begin{snugshade}}{\end{snugshade}}
\newcommand{\AlertTok}[1]{\textcolor[rgb]{0.94,0.16,0.16}{#1}}
\newcommand{\AnnotationTok}[1]{\textcolor[rgb]{0.56,0.35,0.01}{\textbf{\textit{#1}}}}
\newcommand{\AttributeTok}[1]{\textcolor[rgb]{0.77,0.63,0.00}{#1}}
\newcommand{\BaseNTok}[1]{\textcolor[rgb]{0.00,0.00,0.81}{#1}}
\newcommand{\BuiltInTok}[1]{#1}
\newcommand{\CharTok}[1]{\textcolor[rgb]{0.31,0.60,0.02}{#1}}
\newcommand{\CommentTok}[1]{\textcolor[rgb]{0.56,0.35,0.01}{\textit{#1}}}
\newcommand{\CommentVarTok}[1]{\textcolor[rgb]{0.56,0.35,0.01}{\textbf{\textit{#1}}}}
\newcommand{\ConstantTok}[1]{\textcolor[rgb]{0.00,0.00,0.00}{#1}}
\newcommand{\ControlFlowTok}[1]{\textcolor[rgb]{0.13,0.29,0.53}{\textbf{#1}}}
\newcommand{\DataTypeTok}[1]{\textcolor[rgb]{0.13,0.29,0.53}{#1}}
\newcommand{\DecValTok}[1]{\textcolor[rgb]{0.00,0.00,0.81}{#1}}
\newcommand{\DocumentationTok}[1]{\textcolor[rgb]{0.56,0.35,0.01}{\textbf{\textit{#1}}}}
\newcommand{\ErrorTok}[1]{\textcolor[rgb]{0.64,0.00,0.00}{\textbf{#1}}}
\newcommand{\ExtensionTok}[1]{#1}
\newcommand{\FloatTok}[1]{\textcolor[rgb]{0.00,0.00,0.81}{#1}}
\newcommand{\FunctionTok}[1]{\textcolor[rgb]{0.00,0.00,0.00}{#1}}
\newcommand{\ImportTok}[1]{#1}
\newcommand{\InformationTok}[1]{\textcolor[rgb]{0.56,0.35,0.01}{\textbf{\textit{#1}}}}
\newcommand{\KeywordTok}[1]{\textcolor[rgb]{0.13,0.29,0.53}{\textbf{#1}}}
\newcommand{\NormalTok}[1]{#1}
\newcommand{\OperatorTok}[1]{\textcolor[rgb]{0.81,0.36,0.00}{\textbf{#1}}}
\newcommand{\OtherTok}[1]{\textcolor[rgb]{0.56,0.35,0.01}{#1}}
\newcommand{\PreprocessorTok}[1]{\textcolor[rgb]{0.56,0.35,0.01}{\textit{#1}}}
\newcommand{\RegionMarkerTok}[1]{#1}
\newcommand{\SpecialCharTok}[1]{\textcolor[rgb]{0.00,0.00,0.00}{#1}}
\newcommand{\SpecialStringTok}[1]{\textcolor[rgb]{0.31,0.60,0.02}{#1}}
\newcommand{\StringTok}[1]{\textcolor[rgb]{0.31,0.60,0.02}{#1}}
\newcommand{\VariableTok}[1]{\textcolor[rgb]{0.00,0.00,0.00}{#1}}
\newcommand{\VerbatimStringTok}[1]{\textcolor[rgb]{0.31,0.60,0.02}{#1}}
\newcommand{\WarningTok}[1]{\textcolor[rgb]{0.56,0.35,0.01}{\textbf{\textit{#1}}}}
\usepackage{graphicx,grffile}
\makeatletter
\def\maxwidth{\ifdim\Gin@nat@width>\linewidth\linewidth\else\Gin@nat@width\fi}
\def\maxheight{\ifdim\Gin@nat@height>\textheight\textheight\else\Gin@nat@height\fi}
\makeatother
% Scale images if necessary, so that they will not overflow the page
% margins by default, and it is still possible to overwrite the defaults
% using explicit options in \includegraphics[width, height, ...]{}
\setkeys{Gin}{width=\maxwidth,height=\maxheight,keepaspectratio}
% Set default figure placement to htbp
\makeatletter
\def\fps@figure{htbp}
\makeatother
\setlength{\emergencystretch}{3em} % prevent overfull lines
\providecommand{\tightlist}{%
  \setlength{\itemsep}{0pt}\setlength{\parskip}{0pt}}
\setcounter{secnumdepth}{-\maxdimen} % remove section numbering

\title{Ejemplo anova de dos vías}
\author{Luis Gerardo Guzmán Rojas}
\date{11/5/2020}

\begin{document}
\maketitle

\hypertarget{almacenamiento-de-datos-en-anova}{%
\subsection{Almacenamiento de datos en
ANOVA}\label{almacenamiento-de-datos-en-anova}}

Vamos a almacenar los datos de una forma parecida a la usada en ANOVA de
un factor.

Sea \(X\) la variable característica de la que comparamos las medias de
las subpoblaciones. Sean \(A\) y \(B\) los factores.

Vamos a transformar la tabla anterior de los datos en una tabla de datos
con \(N=n\cdot a\cdot b\) filas y tres columnas.

La primera columna serán los valores de la variable \(X\), la segunda
los valores o niveles de la variable factor \(A\) y la tercera, los
valores o niveles de la variable factor \(B\).

\hypertarget{ejemplo}{%
\subsection{Ejemplo}\label{ejemplo}}

\begin{example}

La transformación de la tabla de datos para el ejemplo anterior se
realizaría de la forma siguiente:

\begin{Shaded}
\begin{Highlighting}[]
\NormalTok{GSI =}\StringTok{ }\KeywordTok{c}\NormalTok{(}\FloatTok{0.90}\NormalTok{,}\FloatTok{0.83}\NormalTok{,}\FloatTok{1.06}\NormalTok{,}\FloatTok{0.67}\NormalTok{,}\FloatTok{0.98}\NormalTok{,}\FloatTok{0.57}\NormalTok{,}\FloatTok{1.29}\NormalTok{,}\FloatTok{0.47}\NormalTok{,}\FloatTok{1.12}\NormalTok{,}\FloatTok{0.66}\NormalTok{,}
        \FloatTok{1.30}\NormalTok{,}\FloatTok{1.01}\NormalTok{,}\FloatTok{2.88}\NormalTok{,}\FloatTok{1.52}\NormalTok{,}\FloatTok{2.42}\NormalTok{,}\FloatTok{1.02}\NormalTok{,}\FloatTok{2.66}\NormalTok{,}\FloatTok{1.32}\NormalTok{,}\FloatTok{2.94}\NormalTok{,}\FloatTok{1.63}\NormalTok{)}
\NormalTok{temperatura =}\StringTok{ }\KeywordTok{factor}\NormalTok{(}\KeywordTok{rep}\NormalTok{(}\KeywordTok{c}\NormalTok{(}\DecValTok{27}\NormalTok{,}\DecValTok{16}\NormalTok{),}\DataTypeTok{each=}\DecValTok{10}\NormalTok{))}
\NormalTok{fotoperiodos =}\StringTok{ }\KeywordTok{factor}\NormalTok{(}\KeywordTok{rep}\NormalTok{(}\KeywordTok{c}\NormalTok{(}\DecValTok{9}\NormalTok{,}\DecValTok{14}\NormalTok{),}\DataTypeTok{times=}\DecValTok{10}\NormalTok{))}
\NormalTok{tabla.datos.GSI =}\StringTok{ }\KeywordTok{data.frame}\NormalTok{(GSI,temperatura,fotoperiodos)}
\KeywordTok{head}\NormalTok{(tabla.datos.GSI)}
\end{Highlighting}
\end{Shaded}

\begin{verbatim}
##    GSI temperatura fotoperiodos
## 1 0.90          27            9
## 2 0.83          27           14
## 3 1.06          27            9
## 4 0.67          27           14
## 5 0.98          27            9
## 6 0.57          27           14
\end{verbatim}

\end{example}

\end{document}
